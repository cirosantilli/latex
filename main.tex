\documentclass[12pt]{article}

\title{Ciro's latex cheatsheet}
\author{Ciro Duran Santilli}
%\date{01/01/2000} % if commented out, uses today's date

\usepackage{main}

%\newcommand{\inOut}[1]{\begin{verbatim}#1\end{verbatim}#1} %CANT use verbatim here, give up, people will read at source for now!!!
\newcommand{\inOut}[1]{#1}

\begin{document}

%\begin{comment}

\begin{remark}

Put TODOs as the first \label{TODO2} thing for development so you don't forget to do them.

You may want to put those in order of urgency/difficulty here.

Mark TODO location in \label{TODO1} the middle of text with labels TODO, then explain them here.

Comment them out for release.

\end{remark}

\begin{itemize}
  \item TODO2 deal with this TODO at all costs
  \item TODO1 this TODO is not as important as the first one.
  \item TODO3
\end{itemize}

\newpage

%\end{comment}

%\maketitle
%\newpage

\tableofcontents
\newpage

\section{Tables}

\begin{example} \label{expTab1}
A simple table

\inOut{
\begin{table}[h]
  \centering
  \begin{tabular}{ccc}
    1 & 2 & 3 \\
    4 & 5 & 6 \\
    7 & 8 & 9 \\
  \end{tabular}
  \caption{caption}
  \label{tab1}
\end{table}
}

\end{example}\hrule

\begin{remark} \label{remTab1}
  The [h] means that the table should stay at current position (here), and not float around the page if possible.
\end{remark}\hrule

\begin{remark} \label{remTab2}
  For complex tables with label LABEL, create a LABEL.ods spreadsheet with same name as the label and use it to make the table, then copy paste to the .tex.
\end{remark}\hrule

\section{Comments}

Use the comment environment from the verbatim package.

\begin{example} \label{expCom1}

Next line will be commented out.
\begin{comment}
This line was commented off.
\end{comment}
\end{example}
\hrule

\section{References}

\begin{remark} \label{remLab1}
\textbackslash{}label refers to the smallest surrounding thing that is numbered, typically a section or a theorem environment. Therefore, if you simply put a label in a point paragraph, you don't normally get a link to that point of paragraph like you would in an html anchor.
\end{remark}\hrule

\begin{remark} \label{remLab2}
  For url compatibility hyphen separation is a good idea: def-a-definition-was-here.
\end{remark}\hrule

\begin{example} \label{expRef1}
\inOut{
The next ref aims at this example.

Ref: \ref{expRef1}
}
\end{example}

\section{Citations}

\begin{example} \label{expCite1}
\inOut{One \cite{LL00} Two \cite{LL01}}
\end{example}\hrule

\begin{remark} \label{remCite1}
  You have to cite a reference before it appears in the bibliography
\end{remark}\hrule

\section{Bibliography}

In the bibliography command, use the same name as your .bib file.

\newpage

\bibliography{main}

\end{document}